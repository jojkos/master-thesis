% Tento soubor nahraďte vlastním souborem s přílohami (nadpisy níže jsou pouze pro příklad)
% This file should be replaced with your file with an appendices (headings below are examples only)

% Umístění obsahu paměťového média do příloh je vhodné konzultovat s vedoucím
% Placing of table of contents of the memory media here should be consulted with a supervisor
%\chapter{Obsah přiloženého paměťového média}

%\chapter{Manuál}

%\chapter{Konfigurační soubor} % Configuration file

%\chapter{RelaxNG Schéma konfiguračního souboru} % Scheme of RelaxNG configuration file

%\chapter{Plakát} % poster

\chapter{Poznámky} \label{kapitola:poznamky}

\section{TODOs}
http://www.statmt.org/moses/?n=Moses.Releases \ref{moses}

data bych doporucil open subtitles http://opus.lingfil.uu.se/

subtitles projet aligmentem.. oscorovat pary vet a pak vyfiltrovat malo pravdepodobne -> rozumna data\\
stahnuto, zkusil jsem projet moses => po 24 hodinach zaplnilo cely disk a nebylo hotovo. Filtrovani pomoci  -score-options="-MinScore?

wmt vysledky http://www.statmt.org/wmt17/results.html ? BLEU


\section{Pseudo zadani}

chapu to tak, ze: nejdriv projedu kazde slovo na vstupu skrz prepripraveny nauceny veci od facebooku do embeddings a to pak teprve posilam do encoderu. Ten mi z toho pak vyhodi context (thought) vector\\
pouziju bidirectional LSTM (RNN) + attention + beam search\\
kazdej vstup (veta) je zarovnana (padding) na nejakou delku (bud se to doplni nebo naopak oreze), potom se z toho udela embeddings s pouzitim pretrained, potom se prozene encoderem, ziskam context, ten se prozedene dekoderem (s pomoci attention)\\
src input bude reversed sequence, protoze to podle nekterych clanku funguje lip, nevim jestli nestaci pouzit bi-directional LSTM\\

podle thesis je mozny pouzit hybridni model, generovat znama slova po slovech (ne po jednotlivych characterech) a neznama slova po jednotlivych znacich (takze vystupem muze byt o slovo mimo slovnik ze kteryho se sit ucila)\\

myslel jsem ze vystupem je embeding, ze kteryho se napr. podle vzdalenosti zjisti vysledne slovo, ale mozna je spis vystupem one hot encoding velikosti output slovniku a embeddings se pouzivaji jenom v prubeznych vrstvach

\section{Kapitoly}
\begin{itemize}
  \item what is machine learning
  \item machine learning druhy (odvozovani ze znalosti, feature/representation, deep learning)
  \item popis ruznych neuronovych siti - CNN (images), structural?/standard NN (?), RNN(audio/translation), LSTM (translation)?,
  \item hyperparameters
  \item popis ruznych zpusobu strojoveho prekladu textu
  \item popis frameworku? na neuronky
  \item Tensors
  \item activation functions
  \item Tools - moses
  \item frameworks - google tensorflow, microsoft CNTK, theano, keras (with usage of tensorflow/cntk/theano), tf.contrib.Keras + tf
  \item pretrained embeddings - facebook, word2vec, glove
\end{itemize}


\section{Facebook pretrained word vectors}
obsahuji textovou verzi - slovo a jeho vector a binarni verzi ve formatu fastText
\url{https://github.com/facebookresearch/fastText/blob/master/pretrained-vectors.md}\\
\url{https://blog.manash.me/how-to-use-pre-trained-word-vectors-from-facebooks-fasttext-a71e6d55f27}\\
\url{https://www.quora.com/What-is-the-main-difference-between-word2vec-and-fastText}

\section{Moses}
\begin{itemize} \label{moses}
  \item UKAZKA JAK TO POUZIVA GOOGLE TUTORIAL \url{https://github.com/tensorflow/nmt/blob/master/nmt/scripts/wmt16_en_de.sh}
  \item statistical machine translation (SMT) or probably syntax based translation or factored translation
  \item data preparation
  \begin{itemize}
    \item tokenisation: This means that spaces have to be inserted between (e.g.) words and punctuation.
    \item truecasing: The initial words in each sentence are converted to their most probable casing. This helps reduce data sparsity.
    \item cleaning: Long sentences and empty sentences are removed as they can cause problems with the training pipeline, and obviously mis-aligned sentences are removed.
  \end{itemize}
  \item co zatim zkousim, rucne postupne jednotlive kroky
  \begin{itemize}
    \item mam data z http://opus.lingfil.uu.se/ pro cs (Czech)/en (English) pro moses, tzn tri soubory moses.cs-en.cs, moses.cs-en.en, moses.cs-en.ids
    \item tokenizace
    \begin{lstlisting}
    ~/mosesdecoder/scripts/tokenizer/tokenizer.perl -l en \
    < /media/sf_DPbigFiles/OpenSubtitles2016-moses.cs-en.en \
    > /media/sf_DPbigFiles/OpenSubtitles2016-moses.cs-en.tokenized.en
    \end{lstlisting}
    \item nauceni truecaser modelu
    \begin{lstlisting}
    ~/mosesdecoder/scripts/recaser/train-truecaser.perl \
    --model /media/sf_DPbigFiles/truecase-model.en \
    --corpus /media/sf_DPbigFiles/OpenSubtitles2016-moses.cs-en.tokenized.en
    \end{lstlisting}
    \item truecased
    \begin{lstlisting}
    ~/mosesdecoder/scripts/recaser/truecase.perl \
    --model /media/sf_DPbigFiles/truecase-model.en \
    < /media/sf_DPbigFiles/OpenSubtitles2016-moses.cs-en.tokenized.en \
    > /media/sf_DPbigFiles/OpenSubtitles2016-moses.cs-en.tokenized.truecased.en
    \end{lstlisting}
    \item cleaning
    \begin{lstlisting}
    ~/mosesdecoder/scripts/training/clean-corpus-n.perl /media/sf_DPbigFiles/OpenSubtitles2016-moses.cs-en.tokenized.truecased cs en /media/sf_DPbigFiles/OpenSubtitles2016-moses.cs-en.tokenized.truecased.cleaned 1 80
    \end{lstlisting}
    \item language model training
    \begin{lstlisting}
    /media/sf_DPbigFiles/languagemodel$ ~/mosesdecoder/bin/lmplz -o 3 < /media/sf_DPbigFiles/OpenSubtitles2016-moses.cs-en.tokenized.truecased.cleaned.en > OpenSubtitles2016-moses.cs-en.arpa.en
    \end{lstlisting}
    \item binarizing for faster loading
    \begin{lstlisting}
    ~/mosesdecoder/bin/build_binary OpenSubtitles2016-moses.cs-en.arpa.en OpenSubtitles2016-moses.cs-en.binary.en
    \end{lstlisting}
    \item training - ZKUSIT PUSTIT BEZ PARAMETRU rika to pak ty jednotlivy stepy co chci udelat
    \begin{lstlisting}
    ~/mosesdecoder/scripts/training/train-model.perl -root-dir . --corpus /media/sf_DPbigFiles/OpenSubtitles2016-moses.cs-en.tokenized.truecased.cleaned --f cs --e en -external-bin-dir ~/mosesdecoder/tools/ -cores 4 -parallel -lm 0:3:/media/sf_DPbigFiles/languagemodel/OpenSubtitles2016-moses.cs-en.binary.en
    \end{lstlisting}
    \begin{lstlisting}
    with mgiza++
    ~/mosesdecoder/scripts/training/train-model.perl -root-dir train -corpus ~/corpus/news-commentary-v8.fr-en.clean -f fr -e en -alignment grow-diag-final-
    and -reordering msd-bidirectional-fe -lm 0:3:$HOME/lm/news-commentary-v8.fr-en.blm.en:8 -external-bin-dir ~/mosesdecoder/tools -cores 4 -parallel -mgiza -mgiza-cpus 4
    \end{lstlisting}
  \end{itemize}
  OpenSubtitles2016-moses.cs-en.cs/en jsou moc velky, zkusim vzit mensi cast (prvnich x radku) a natrenovat to s tim. Puvodni velka obsahuje 33896950 radku.

  \begin{itemize}
    \item pouziti EMS - Experiment Management System, ktery obdrzi konfiguracni soubor a resi si jednotlive kroky a skripty sam
    \item nainstaloval jsem xming, potreba pred spustenim skriptu "export DISPLAY=:0"
    \item ~/mosesdecoder/scripts/ems/experiment.perl -config config.toy -exec
    \item spadlo to na step EVALUATION:test:nist-bleu crashed step EVALUATION:test:nist-bleu-c crashed
    \item takze asi zustanu u rucne postupnych prikazu - vytvoren vlastni skrupt runAll.sh. bash -x ./runAll.sh \url{http://www.statmt.org/moses/?n=Moses.Baseline}
  \end{itemize}
\end{itemize}

\section{odkazy}
\url{https://en.wikipedia.org/wiki/Language_model}

\begin{itemize}
    \item RBMT \url{https://en.wikipedia.org/wiki/Rule-based_machine_translation}
    \item SMT \url{https://en.wikipedia.org/wiki/Statistical_machine_translation}
    \item nejaky dalsi...?
    \item neuronka (GNMT, Transfomer)
\end{itemize}

computing BLEU score in python using ntlk lib \url{http://www.nltk.org/_modules/nltk/align/bleu.html}

research at google - machine translation articles \\ \url{https://research.google.com/pubs/MachineTranslation.html}\\
\url{https://en.wikipedia.org/wiki/Google_Neural_Machine_Translation}\\

\url{https://research.googleblog.com/2016/09/a-neural-network-for-machine.html}\\

\url{https://research.googleblog.com/2017/06/accelerating-deep-learning-research.html} \\
\url{https://research.googleblog.com/2017/07/building-your-own-neural-machine.html} \\
\url{https://research.googleblog.com/2017/04/introducing-tf-seq2seq-open-source.html}\\

\textbf{BUCKETING AND PADDING IN TENSOR FLOW}
\url{https://www.tensorflow.org/tutorials/seq2seq#bucketing_and_padding}

\textbf{neural machine translation tutorial acl 2016}
\url{https://sites.google.com/site/acl16nmt/home}

\textbf{chat bot in keras}
\url{https://github.com/saurabhmathur96/Neural-Chatbot}

\url{https://en.wikipedia.org/wiki/Language_model}

\section{knihovny nad tensorflow}
\begin{itemize}
  \item \url{google.github.io/seq2seq} - A general-purpose encoder-decoder framework for Tensorflow, pomoci konfiguraci, snadne vytvoreni komplexnich seq2seq modelu, pouzity pro clanek Massive Exploration of Neural Machine Translation Architectures. NENI UDRZOVANO z https://gitter.im/tensor2tensor/Lobby -  google/seq2seq is not maintained. If you're just starting, read this first https://github.com/tensorflow/nmt and read papers, the one for this repo too.
  \item \url{github.com/tensorflow/tensor2tensor}- A library for generalized sequence to sequence models, pomoci konfiguraci (vyber ruznych modulu a moznost vytvoreni vlastni), asi relativne snadny vytvoreni ruznych (obecnych, nejen seq2seq) modelu. VYPADA TO ROZUMNE
  \item \url{github.com/tensorflow/nmt} - zatimco T2T uz je hlavne skladacka predpripravenych veci, tenhle tutorial ukazuje jak vyrobit v tensorflow od pocatku vlastni model
\end{itemize}


\section {Neural Machine Translation (seq2seq) Tutorial}
\url{https://github.com/tensorflow/nmt} podle \url{https://github.com/lmthang/thesis}, uzitecnej clanek

\begin{itemize}
  \item nmt model can differ in terms of \emph{directionality} uni/bi, \emph{depth} single/multi layer, \emph{type} vanilla RNN/LSTM/GRU
  \item In this tutorial, we consider as examples a deep multi-layer RNN which is unidirectional and uses LSTM as a recurrent unit.
  \item time-major format znamena ze prvni parameter je max\_encoder\_time a druhy batchsize, u batch-major je to naopak
  \item DECODER teda funguje tak, ze dostava 1. vysledek z encoderu, tim vi z ceho preklada a k tomu 2. nejdriv znak <s> pro zacatek dekodovani a v dalsich casovych stepech pak pri treningu ty spravne slova prekladu a pri pouziti pak ty slova co sam vygeneruje. Je to dobre popsany v \url{https://github.com/tensorflow/nmt#inference--how-to-generate-translations}
  \item ukazka z tutorialu v gitbashi spadne s encoding problemem, v cmd ne
  \item ukazka spadne na nedostatku pameti, je potreba zmenit parametry. (po kazde zmene radsi smazat slozku nmt\_model).
      \begin{lstlisting}
pro cmd
python -m nmt.nmt ^
    --src=vi --tgt=en ^
    --vocab_prefix=nmt_data/vocab  ^
    --train_prefix=nmt_data/train ^
    --dev_prefix=nmt_data/tst2012  ^
    --test_prefix=nmt_data/tst2013 ^
    --out_dir=nmt_model ^
    --num_train_steps=1000 ^
    --steps_per_stats=100 ^
    --num_layers=2 ^
    --num_units=32 ^
    --batch_size=64 ^
    --dropout=0.2 ^
    --metrics=bleu
    \end{lstlisting}
vyzkousim jak to funguje
\begin{lstlisting}
python -m nmt.nmt ^
    --out_dir=nmt_model ^
    --inference_input_file=nmt_data/my_infer_file.vi ^
    --inference_output_file=nmt_model/output_infer
\end{lstlisting}
\end{itemize}

\section{old tensorflow seq2seq tutorial}
\url{https://www.tensorflow.org/tutorials/seq2seq}
\begin{lstlisting}
old seq2seq tutorial G:\Dropbox\vecicky\python\tensorFlow\RNNtutorial\translate
python translate.py ^
  --data_dir=nmt_data --train_dir=train ^
  --from_vocab_size=100 --to_vocab_size=100 ^
  --from_train_data=nmt_data/OpenSubtitles2016-moses-10000.cs-en-tokenized.truecased.cleaned.cs ^
  --to_train_data=nmt_data/OpenSubtitles2016-moses-10000.cs-en-tokenized.truecased.cleaned.en ^
  --size=256 --batch_size=32 --num_layers=1

spadlo to zas na pameti, zkusim mensi size a tak

https://github.com/tensorflow/tensorflow/issues/11157
I suffered exactly the same problem.
I just passed the error by modifying 2 lines of the following seq2seq.py file from Tensorflow.

file: Anaconda3\Lib\site-packages\tensorflow\contrib\legacy_seq2seq\python\ops\seq2seq.py
848 #encoder_cell = copy.deepcopy(cell)
849 encoder_cell = core_rnn_cell.EmbeddingWrapper(
850 cell, #encoder_cell,

preklad
python translate.py --decode --data_dir=nmt_data --train_dir=train ^
    --from_vocab_size=100 --to_vocab_size=100
\end{lstlisting}

\section{Tensor2Tensor}
\url{github.com/tensorflow/tensor2tensor}
\begin{itemize}
  \item po instalaci na windows nefunguji pripravene bin programy podle tutorialu (jako t2t-trainer). Je potreba stahnout si je z repositare a poustet rucne - python t2t-trainer
  \begin{lstlisting}
python bin/t2t-trainer.py ^
  --generate_data ^
  --data_dir=walktrough/t2t_data ^
  --problems=translate_enmk_setimes32k ^
  --model=transformer ^
  --hparams_set=transformer_base_single_gpu ^
  --output_dir=walktrough/t2t_train/base ^
  --hparams="batch_size=64"

  a stejne to spadne s
  InternalError (see above for traceback): Blas SGEMM launch failed : m=90, n=1536, k=512\\
  zmenil jsem flags.DEFINE_float("worker_gpu_memory_fraction", 0.35,
                   "Fraction of GPU memory to allocate.") v souboru knihovny tensor2tensor trainer_utils.py z 0.95 na 0.35 a stejne to spadne na OOM
  \end{lstlisting}
\end{itemize}

\section{transformer, tensorFlow}
\begin{itemize}
    \item googls novel neural architecture - Transformer better than GNMT (google neural machine translation) \url{https://research.googleblog.com/2017/08/transformer-novel-neural-network.html}
    \item tensor2tensor \url{https://github.com/tensorflow/tensor2tensor/}
    \item t2t \url{https://research.googleblog.com/2017/06/accelerating-deep-learning-research.html}
    \item Train set for training, validation set for hyperparameters tuning, test set for testing how good
\end{itemize}

transformer - uses only attention and gets rid of recurrence and convolution. What exactly is and does recurrence and convolution (in context of neural networks)? Probably another thing that could be written in the theoretical part.



\section{vypisky z deep learning book}
\begin{itemize}
    \item uvod a historie, jak se postupne menily a jaky byly ruzny druhy machine learning
    \item machine learning basics
    \begin{itemize}
      \item klasifikace
      \item klasifikace s chybejicimi vstupy
      \item regrese
      \item transription
      \item MACHINE TRANSLATION
      \item structured output
      \item anomaly detection
      \item Synthesis and sampling
      \item Imputation of missing values
      \item Denoising
      \item Density estimationorprobability mass function estimation
    \end{itemize}
    \item Task, Performance measure, Expericen..to co to ma delat, cim a jak se zmeri jak dobre to dela, cinnost na ktere se to nauci
    \item unsupervised - nema popisky a pocitac se snazi sam z dat urcit nejake zavery - treba clustering, supervised - ty maji label pro data v data setu, takze se nauci pro ktere x je jake y a pak se snazi odvodit y pro dalsi nahodne x, ktere se jim predhodi
    \item underfitting, overfitting
    \item train error (chybovost na trainovacim data setu) vs generalization error (chybovost na testovacim datasetu)
    \item regularization
    \item hyperparameters
    \item train vs test vs validation set
    \item regression - ziskavame nejakou hodnotu na zaklade parametru, klasifikace - rozrazujeme do presne danych trid
\end{itemize}

5.4 Estimators, Bias and Variance

\section{coursera deeplearning}
vypisky z \url{https://www.coursera.org/learn/neural-networks-deep-learning}
\begin{itemize}
  \item RELU - rectified linear unit, nahrazuje sigmoid funkci, protoze se nad ni rychleji uci
  \item structured data - tabulky informaci, kazdej sloupec je jedna feature (age, bedooorms, price) X unstructured data - obrazky, hudba, text
  \item logistic regression - jaka je procentualni sance ze x na vstupu = nejake vystupni y (asi jenom jedno konkretni), je to binarni klasifikace. Na rozrazeni do 0-1 pouziva sigmoid funkci
  \item cost function pro logistickou regresi - prumer loss funkce nad celym trenovacim setem
  \item vektor v matici je jako jeden sloupec
  \item (VEKTORIZACE) nasobeni vektoru v maticich misto ve smycce je radove rychlejsi! (numpy.dot), SIMD instrukce (single instruction, multiple data)
  \item broadcasting (v pythonu) vstupni hodnotu (at uz matici nebo skalar) namnozi takovym zpusobem, aby sla pouzit v operaci s matici
  \item The main steps for building a Neural Network are:
  \begin{itemize}
    \item Define the model structure (such as number of input features)
    \item Initialize the model's parameters
    \item Loop:
    \item Calculate current loss (forward propagation)
    \item Calculate current gradient (backward propagation)
    \item Update parameters (gradient descent)
  \end{itemize}
  \item $Z_{n}^{[l](v)}$ l - index of layer (hidden), v - index of training vector, n - index of node in the layer
  \item activation functions (sigmoid 0-1 better for output layer for binary classification, tanh -1 - 1 better for hidden layers, (leaky) RELU 0/1 even better)
  \item bias can be initialized to zero but weights must be initialized randomly because all the nodes inside layer would have the same weights and would be calculating the same numbers (they would be identical)
\end{itemize}

\section{bridging the gap}
vypisky z Google’s Neural Machine Translation System: Bridging the Gap
between Human and Machine Translation
\begin{itemize}
  \item encoder (LSTM RNN) - transforms a source sentence into a list of vectors, one vector per input symbol; 8 layers
  \item decoder (LSTM RNN) - produces on symbol at a time from the vectors; 8 layers
  \item those two are connected through an attention module - feed forward network with one hidden layer
  \item attention - koukne se pro kazde slovo na jeho okoli a pri prekladu se rozhodne, ktera slovo s tim danym slovem nejvice souvisi a podle toho vybere spravny preklad
  \item residual connections enable to train much deeper networks
  \item neural network model weights can be quantizied to speed up some inference
  \item BLEU score metric
  \item pouziti wordpieces (vylepsuje handlig rare slov), coz umoznuje generovani novych slov jako pri pouziti modelu po jednotlivych pismenech, ale je to efektivnejsi jako pri pouziti celych slov
  \item Using wordpieces gives a good balance between the flexibility of single characters and the efficiency of full words for decoding, and also sidesteps the need for special treatment of unknown words.
\end{itemize}

\section{deep learning thesis}
vypisky z thesis(nmtTutorialBasedOnThis) https://github.com/lmthang/thesis
\begin{itemize}
  \item Language modeling is an important concept in natural language processing to allow one to do word prediction, i.e., guessing which word will come next given a preceding context.
  \item it does so by predicting next words in a text given a history of previous words.
  \item word embeddings are used instead of one-hot representation for words (long vector, one value for each word in vocabulary, 0 meaning false and 1 meaning true). Word embeddings has the same meaning value but are much smaller matrices.
\end{itemize}

\section{clanek sequence to sequence learning with nn}
\begin{itemize}
  \item normal deep neural network models are excellent on many task, but not on mapping sequence to sequence.
  \item use LSTM to map input sequence to thought vector (encoder) then decoder to map to target sequence
  \item reversing order of words in all source sentences improves LSTM's performance markedly because of many short term dependencies
  \item Despite their flexibility and power, DNNs can only be applied to problems whose inputs and targets can be sensibly encoded with vectors of fixed dimensionality.
  \item The goal of the LSTM is to estimate the conditional probability
  \item again USES DATASET English to French translation task from the WMT 14 dataset

\end{itemize}

\section{clanek unsupervised machine translation using monolingual corpora only}
\begin{itemize}
  \item model that takes sentences from monolingual corpora in two different languages and maps them into the same latent space
  \item By learning to reconstruct in both languages from this shared feature space, the model effectively learns to translate without using any labeled data.
  \item TWO WIDELY USED DATASETS and two language pairs - zkusit zjistit ktery jsou widely used datasets a pouzit je taky
  \item the model has to be able to reconstruct a sentence in a given language from a noisy version of it, as in standard denoising auto-encoders
  \item The model also learns to reconstruct any source sentence given a noisy translation of the same sentence in the target domain, and vice versa.
  \item jak vyuziva decoder word embeddings corresponddujici k danemu jazyku?
  \item The encoder is a bidirectional-LSTM which returns a sequence of hidden states z. At each step, the decoder, which is an LSTM, takes the previous hidden state, the current word and a context vector given by a weighted sum over the encoder states.
  \item jak funguje naivni inicializace s unsporvised vytvorenym word by word prekladem?
  \item chapu to tak, ze preklad funguje nasledovne - vezme se source veta v source jazyce, prelozi se translation modelem M (ktery je, nevim co?) a vznikne tak ne uplne povedeny preklad. Protoze se to predtim ucilo autoencodovat z do stejneho jazyka na poskozenych vetach, tak je nasledovne mozne tento poskozeny preklad prohnat autoencoderem a tim dostat spravny preklad (protoze se to predtim ucilo z pozkozenych vet tvorit spravne vety).
  \item M je nazacatku unsupervised word-by-word translation model using the inferred dictionary
  \item pouzivaji WMT 14 English-French a WMT 16 English-German
  \item \url{http://www.statmt.org/wmt14/translation-task.html}
\end{itemize}

\section{clanek Learning Phrase Representations using RNN Encoder–Decoder for Statistical Machine Translation}
\begin{itemize}
  \item The encoder maps a variable-length source sequence to a fixed-length vector, and the decoder maps the vector representation back to a variable-length target sequence.
  \item The two networks are trained jointly to maximize the conditional probability of the target sequence given a source sequence.
  \item RNN Encoder–Decoder learns a continuous space representation of a phrase that preserves both the semantic and syntactic structure of the phrase.
  \item RNN is neural network with hidden state h and optional output y which takes variable length input x. $h_t = f(h_{t-1}, x_t)$
  \item After reading the end of the sequence (marked by an end-of-sequence symbol), the hidden state of the RNN is a summary c of the whole input sequence.
  \item baseline model - popsany co jak vybraly za data, WMT14 english-french, SMT system v moses s default settings
\end{itemize}

\url{http://colah.github.io/posts/2015-08-Understanding-LSTMs/}
\section{understanding LSTM networks and }\label{LSTM}
\begin{itemize}
  \item potrebuju pochopit co je vystup encoderu/LSTM, rozdil mezi hidden state a cell(memory) state a co presne vechno se pak z toho pouzije v decoderu viz clanek predtim
  \item url{https://www.quora.com/What-is-the-difference-between-states-and-outputs-in-LSTM}
  \item zakladni RNN si neumi pamatovat veci pres delsi casovej usek (single tanh layer)
  \item LSTM je se reseni - STMs are explicitly designed to avoid the long-term dependency problem
  \item CELL STATE
  \begin{itemize}
    \item The cell state is kind of like a conveyor belt. It runs straight down the entire chain, with only some minor linear interactions. It’s very easy for information to just flow along it unchanged.
    \item The LSTM does have the ability to remove or add information to the cell state, carefully regulated by structures called gates.
    \item Gates are a way to optionally let information through. They are composed out of a sigmoid neural net layer and a pointwise multiplication operation.
    \item The sigmoid layer outputs numbers between zero and one, describing how much of each component should be let through. A value of zero means “let nothing through,” while a value of one means “let everything through!” An LSTM has three of these gates, to protect and control the cell state.
    \item FIRST forget values based on $W_f$, then get learn new values based on $W_i and W_C$ and finally get new cell state from it
  \end{itemize}
  \item OUTPUT $h_t$ is based on cell state and filtered and shifted to -1 and 1 values using another weight $W_o$
  \item pochopil jsem cell state jako stav, ke kteremu se dojde patrne jednim pruchodem/prubehem/iteraci zkrz LSTM vrstvu (takze treba kdyz do toho poslu jednu sequenci), ve kterem se/jaky bunka ma na konci. tzn dostalo to nakou vetu a postupne si to z ni neco bralo a zapominalo a na konci to ma ve svym cell state
  \item zatimco hidden state je output (ktery v pripade encoderu nevim co je)
  \item kde jsou v LSTM vahy ktery se uci? patrne uvnitr tech jednotlivych gate a urcuji prave co si to prenasi mezi krokama sekvence INPUT/OUTPUT/FORGET gate. vaha pro forget layer, pro input gate layer a candidate values
  \item OUTPUT is The vector of outputs from all memory units is the output of the LSTM network.

\end{itemize}

\section{The Unreasonable Effectiveness of Recurrent Neural Networks}
\url{http://karpathy.github.io/2015/05/21/rnn-effectiveness/}
\begin{itemize}
  \item If training vanilla neural nets is optimization over functions, training recurrent nets is optimization over programs.
  \item it is known that RNNs are Turing-Complete in the sense that they can to simulate arbitrary programs (with proper weights).
  \item vanilla RNN (nebo rekneme spis obecne RNN), jeden krok je pronasobeni vah W s vnitrnim hidden stavem h, ten se updatuje kazdy krok pomoci nejake funkce (tanh, v lstm je tam zapominaci a ucici se gate..) a vystupem teda je nasobek W*h
  \item character-level language model: That is, we’ll give the RNN a huge chunk of text and ask it to model the probability distribution of the next character in the sequence given a sequence of previous characters. This will then allow us to generate new text one character at a time.
  \item docela hezky popsana backpropagace v rnn u obrazku s prikladem "hello"
  \item
\end{itemize}

\section{Language models}
\url{https://machinelearningmastery.com/statistical-language-modeling-and-neural-language-models/}
\begin{itemize}
  \item Language modeling is the task of assigning a probability to sentences in a language. […] Besides assigning a probability to each sequence of words, the language models also assigns a probability for the likelihood of a given word (or a sequence of words) to follow a sequence of words (Page 105 Neural Network Methods in Natural Language Processing, 2017.)
  \item Language modeling is the art of determining the probability of a sequence of words. This is useful in a large variety of areas including speech recognition, optical character recognition, handwriting recognition, machine translation, and spelling correction
  \item The use of neural networks in language modeling is often called Neural Language Modeling, or NLM for short.
  \item Neural Language Models (NLM) address the n-gram data sparsity issue through parameterization of words as vectors (word embeddings) and using them as inputs to a neural network. The parameters are learned as part of the training process. Word embeddings obtained through NLMs exhibit the property whereby semantically close words are likewise close in the induced vector space.
  \item The neural network approach to language modeling can be described using the three following model properties, taken from “A Neural Probabilistic Language Model“, 2003.
      Associate each word in the vocabulary with a distributed word feature vector. Express the joint probability function of word sequences in terms of the feature vectors of these words in the sequence. Learn simultaneously the word feature vector and the parameters of the probability function.
\end{itemize}

\section{clanek Neural Machine Translation and Sequence-to-sequence Models:A Tutorial}
\begin{itemize}
  \item n gram models - The parameters of n gram models consist of probabilities of the next word given n 1 previous words
  \item has definition of training/development/test data (3.3)
  \item definition of perplexity
  \item ngram vs log-linear models - log linear models calculate the same probabilty of word given a context, but use feature vector. Feature function takes a context as input and gives feature vector as output - for example identity of a word is a vector that is why all words has unique id. One hot vectors are used
  \item dobrej popis LSTM a celkove RNN
  \item residual connections aby se vyhlo vanishing gradientu
  \item popis batchingu a vyhod ruznych velikosti
  \item muzu v kerasu posilat ruzny delky nebo musim mit vsechny stejne dlouhy s paddingem? to je prece plytvani! mohl bych je rozdelit na ruzny velikosti a volat fit takhle s ruznyma, abych to optimalizoval DNESKA- BUCKETING v tensorflow
  \item dobrej popis encoder-decoder, pouzil bych vzdycky odkaz na puvodni zdroj a podle tohodle pak psal, protoze je to hezky pochopitelny
  \item pouziti beam search misto argmax, ruzny encodovani a ensembling?
  \item vyzkouset nebo popsat proc reverse vstupu nebo zkusit misto toho BI-DIRECTIONAL encoder
  \item takze uz mam do budoucna bidirectional encoder, beam search, bucketing, rare (unknown/oov words)
  \item popsat ruzny optimizery adam/gradient..
\end{itemize}

\section{TENSORFLOW}
\begin{itemize}
  \item NVIDIA GTX 760, CUDA support + cudnn
  \item tensorboard
  \item it is often needed to reduce batch size/unit count, because otherwise there is an resourceExhausted error (memory on GPU is too low?)
\end{itemize}

\section{keras}
\url{https://blog.keras.io/a-ten-minute-introduction-to-sequence-to-sequence-learning-in-keras.html}
detailni rozbor keras blogu
\url{https://machinelearningmastery.com/define-encoder-decoder-sequence-sequence-model-neural-machine-translation-keras/}
\begin{itemize}
  \item pouziva preklad po znacich misto po slovech
  \item LSTM in keras and time distributed layer \url{https://machinelearningmastery.com/timedistributed-layer-for-long-short-term-memory-networks-in-python/}
  \item inference mode as opposed to learning mode where we insert into decoder start tag and the whole correct translation:
    1) Encode the input sequence into state vectors.
    2) Start with a target sequence of size 1 (just the start-of-sequence character).
    3) Feed the state vectors and 1-char target sequence to the decoder to produce predictions for the next character.
    4) Sample the next character using these predictions (we simply use argmax).
    5) Append the sampled character to the target sequence
    6) Repeat until we generate the end-of-sequence character or we hit the character limit.
    \item A Dense output layer is used to predict each character. This Dense is used to produce each character in the output sequence in a one-shot manner, rather than recursively, at least during training. This is because the entire target sequence required for input to the model is known during training. The Dense does not need to be wrapped in a TimeDistributed layer.
    \item we can plot the model to file using keras plot-model, graphviz must be installed (both through pip in python and in windows as binary)
    \item pro musi byt padding a fixed length? \url{https://danijar.com/variable-sequence-lengths-in-tensorflow/}, ale jak teda muze google prekladat libovolne dlouhy vety? rozdeli je na mensi?
    \item c - cell state, h - hidden state, vysvetleno lip v section \ref{LSTM} \url{https://machinelearningmastery.com/return-sequences-and-return-states-for-lstms-in-keras/}
    \item \url{https://stackoverflow.com/questions/44515336/how-do-i-show-both-training-loss-and-validation-loss-on-the-same-graph-in-tensor}
    \item \textbf{WARNING: }in python set(chars) doesn't return the same set everytime, order isnt given in set!! must sort as a list, otherwise on python close, the loaded model weights wouldn't correspond correctly to it!!!!
    \item LSTM stateful \url{https://stackoverflow.com/a/46331227}, zaver - defaultni stateful false je v pohode a kazda sequence ma vlastni novej C state
    \item automatic early stopping based on val\_loss \url{https://stackoverflow.com/questions/43906048/keras-early-stopping}
\end{itemize}

\textbf{embeddings in keras}
\url{https://machinelearningmastery.com/use-word-embedding-layers-deep-learning-keras/}
\begin{itemize}
  \item A word embedding is a class of approaches for representing words and documents using a dense vector representation.
  \item The position of a word within the vector space is learned from text and is based on the words that surround the word when it is used.
  \item The position of a word in the learned vector space is referred to as its embedding.
  \item The keras embedding layer requires that the input data be integer encoded, so that each word is represented by a unique integer. This data preparation step can be performed using the Tokenizer API also provided with Keras.
  \item similar to \url{https://blog.keras.io/using-pre-trained-word-embeddings-in-a-keras-model.html}
  \item embedding UNK a ZERO v \url{https://chunml.github.io/ChunML.github.io/project/Sequence-To-Sequence/}
  \item embeddings zobrazitelne v tensorboard pomoci tensorboard callbecku s embeddings freq a embeddings\_metadata souboru s metadaty (teoreticky jde pouzit pretrained embedings soubor, ale asi bude praktictejsi vzit z neho jen pouzitej slovnik, kvuli performance). viz slozka machinelearning mastery s embedingsMetadata.txt
\end{itemize}

attention in keras
\url{https://github.com/philipperemy/keras-attention-mechanism}

\url{https://chunml.github.io/ChunML.github.io/project/Creating-Text-Generator-Using-Recurrent-Neural-Network/}
\url{https://chunml.github.io/ChunML.github.io/project/Sequence-To-Sequence/}
\begin{itemize}
  \item
\end{itemize}

\section{numpy}
indexing and slicing \url{https://stackoverflow.com/questions/2725750/slicing-arrays-in-numpy-scipy}

\section{leany a herout}
\begin{itemize}
  \item udelat comics verzi
  \item V textu používejte autorské "my", "já" použijte v úvodu a v závěru
  \item beran pise (a heroutovi se my taky nelibi) -- nepouzivejte MY, "testy byly provedeny" namísto "my jsme provedli testy"
  \item takze proste v uvodu a zaberu subjektivni veci dam s ja, jinak ne. my taky ne a popisu to nejakym jinym zpusobem
  \item Úvod je úvod k textu diplomky, ne úvod do problematiky. to je az nasledujici teoreticka cast
  \item Názvy kapitol ať přesně a jednoznačně vystihují, co kapitola obsahuje. spatne-teorie, detekce, navrh reseni. Dobre- Detekce objektů příznakovými klasifikátory..
  \item pouzivat prvni osobu (mnoznou nebo jednotnou) jen kdyz pisu o necem co jsem udelal nebo se me tyka. Obecny veci a fakt je jestli to chapu lepsi psat jinak - Kdyz se podivame na vysledky (spatne) / Z vysledku vyplyva (dobre?)
  \item nepouzivat osloveni ctenare (vy..) - Podivejte se na obrazek (spatne), Obrazek ukazuje (dobre)
  \item Pište svou diplomku pro studenta, který má na vaše dílo navázat. \url{http://www.herout.net/blog/2016/04/komu-se-pise-diplomka/}
  \item obsah se musi vejit na jednu stranku
  \item struktura nadpisu by mela mit tri urovne, ctvrta uroven je vetsinou spatne
  \item lověk, který se v oboru aspoň letmo orientuje, přesně poznat, co se v práci nachází. Dokáže odhadnout, co je cílem práce. Ví, z jakých modulů se celé řešení skládá a k čemu tyto slouží. Řekne, kolik a jakých experimentů řešitel provedl. Dokáže říct, kdo je cílovým „zákazníkem“ práce – komu a k čemu je dobrá.
  \item pozor na pomlcky a spojovnik. pomlcka je misto carky a je dlouha (--), spojovnik je napr rikam-li, takze vetsinou chci dve carky za sebou
  \item kazda veta ma sloveso
  \item nezapomenout na uvody kapitol, kde se popisuje strucne a jasne o cem bude
\end{itemize}